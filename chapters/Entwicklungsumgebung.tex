\section{Entwicklungsumgebung}
\subsection{Software}
\begin{table}[htbp]
    \begin{tabu} to \textwidth {| X | X |}
    \hline
    Betriebssysteme & Ubuntu \\
    \hline
    Textverarbeitung & \LaTeX \\
    \hline
    TeX-Distribution & TexLive \\
    \hline
    \LaTeX{-Editor} & Overleaf \\
    \hline
    \gls{UML}-Tool & Umlet\\
    \hline
    Design von \gls{Frontend}/\gls{Backend} & Notability \\
    \hline
    \gls{Python} Editor & Visual Studio Code \\
    \hline
    \gls{HTML} Editor & Visual Studio Code \\
    \hline
    \gls{JavaScript} Editor & Visual Studio Code \\
    \hline
    \gls{Versionskontrolle} & Git \\
    \hline
    Git \gls{Client} & Fork git-cli Visual Studio Code\\
    \hline
    \gls{Remote Repository} mit \gls{Continuous Integration} & Github (github.com) \\
    \hline
    Webserver Frontend/Backend & Django \\
    \hline
    Webserver \gls{REST-API} & Django \\
    \hline
    Skriptsprache für Webanwendungen & python \\
    \hline
    \gls{Datenbank} & postgresql \\
    \hline
    Datenbankadministration & TODO \\
    \hline
    Python-Paketmanager & pip \\
    \hline
    \gls{Webframework} Frontend/Backend & Django \\
    \hline
    Kommunikation im Team & Whatsapp Matrix Discord\\
    \hline
    \glslink{TLS}{TLS} \gls{Certificate Authority} & Let's Encrypt \\
    \hline
\end{tabu}
\end{table}

\subsection{Hardware}
Die Entwicklungsumgebung wird lokal unter Windows, MacOS und Linux entwickelt, jedoch wird es in der Produktion nur auf Linux Servern laufen. Die Systemanforderungen werden möglichst gering gehalten um mögliche Hardware Änderungen ohne Umstellungen vornehmen zu können.
Weitere Infrastruktur wird für den Einsatz von \name nicht benötigt.