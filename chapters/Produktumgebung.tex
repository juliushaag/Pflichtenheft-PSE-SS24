\section{Produktumgebung}
\subsection{Hardware}
Das \gls{Frontend} und \gls{Backend} kann von einem \gls{Server} bereitgesetellt werden, welcher folgende Mindestanforderungen erfüllt:
\begin{itemize}
    \item Ausreichende Rechen- und Speicherkapazität für vorgesehene \gls{Nutzer}zahl, Software und \gls{Datenbank}
    \item Im KIT Netz erreichbar
    \item Zugriffsmöglichkeit auf die Simulationsbenchmarkprogramme Server
\end{itemize}
      
Für den Zugriff auf das Frontend und Backend wird ein Computer mit Betriebssystem benötigt, der an das KIT Netz angeschlossen ist.
\subsection{Software}
\subsubsection{Webdienst}
Der Webdienst von \name benötigt:
\begin{itemize}
    \item Linux-Distribution, vorzugsweise Debian basiert
    \item Relationale Datenbank: postgresql
    \item Python 3
    \item Django mit zugehörigen Abhängigkeiten
\end{itemize}

\subsubsection{Frontend und Backend}
Serverseitig ist für den Betrieb des Frontends und Backends nötig:
\begin{itemize}
    \item Linux-Distribution, vorzugsweise Debian basiert
    \item Apache 2
\end{itemize}
\gls{Client}seitig werden folgende Browser unterstützt:
\begin{itemize}
    \item Safari ab Version 11
    \item Chrome ab Version 69
    \item Firefox ab Version 57
\end{itemize}

\subsection{Orgware}
Ein reibungsloser Betrieb kann nur unter nachstehenden Bedingungen erfolgen:
\begin{itemize}
    \item unterbrechungsfreie Netzwerkverbindung
    \item Verfügbarkeit eines \glslink{Administrator}{Administrators}, der Erstkonfiguration und \gls{Endpunkt}verwaltung im Backend vornimmt
    \item mindestens ein Simulator TODO.
\end{itemize}

