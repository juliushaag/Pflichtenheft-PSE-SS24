\section{Produktdaten}
\label{sec:data}
\setcounter{counter}{10}
\begin{Kriterien}{PD}
\item[\gls{Nutzer}daten] Folgende Nutzerdaten werden gespeichert:
\end{Kriterien}
\vspace{-10pt}
\begin{itemize}\itemsep-10pt
\item Name, Vorname
\item E-Mail-Adresse
\item eindeutige Nutzer-ID
\item vom \gls{LDAP} festgelegte \glspl{Rolle}
\item Ein-/Ausschalten der E-Mail-Benachrichtigung
\end{itemize}


\begin{Kriterien}{PD}
\item[\gls{Administrator}daten] Folgende Administratordaten werden gespeichert:
\end{Kriterien}
\vspace{-10pt}
\begin{itemize}\itemsep-10pt
\item Name, Vorname
\item E-Mail-Adresse
\item Passwort in verschlüsselter Form
\end{itemize}


\begin{Kriterien}{PD}
\item[\gls{Endpunkt}] Für jeden definierten Endpunkt werden folgende Kriterien gespeichert:
\end{Kriterien}
\vspace{-10pt}
\begin{itemize}\itemsep-10pt
\item Name des Endpunkts
\item Eindeutige Endpunkt-ID
\item \gls{Modus} des Endpunkts (parallel/sequentiell)
\item Anzahl der benötigten \glspl{Signatur}
\item Rollen, die signieren müssen
\item Nutzer, die signieren müssen
\item Bei sequentiellem Modus: Reihenfolge der signierenden Rollen/Nutzer
\item Ein-/Ausschalten der E-Mail-Benachrichtigung für Nutzer, deren Signatur benötigt wird
\item \gls{Formular}zusammensetzung
\end{itemize}

\begin{Kriterien}{PD}
\item[\gls{Auftrag}] Jeder Auftrag wird mit folgenden Informationen gespeichert:
\end{Kriterien}
\vspace{-10pt}
\begin{itemize}\itemsep-10pt
\item ID des verwendeten Endpunkts
\item Eindeutige Auftrag-ID
\item Nutzer-ID des Erstellers
\item Erstellungsdatum
\item Benötigte Signaturen
\item Nutzer-ID erfolgter Signaturen mit Zeitstempel
\item Bei sequentiellem Modus: Reihenfolge der Signaturen
\item Ausgefülltes Formular mit zugehörigen Dateien
\end{itemize}

\begin{Kriterien}{PD}
\item[Datei] Für jede hochgeladene Datei werden folgende Metadaten über die Datei gespeichert:
\end{Kriterien}
\vspace{-10pt}
\begin{itemize}\itemsep-10pt
\item Autor
\item Größe
\item Zeitstempel
\item Hashwert
\end{itemize}
